\documentclass{ctexart}

\usepackage{graphicx}
\usepackage{amsmath}
\usepackage{cite}
\usepackage{url}

\title{作业三:我的Linux工作环境}


\author{彭湃3210100061}

\begin{document}

\maketitle

\section{发行版名称以及版本号}

Ubuntu(64-bit) 22.04

\section{已经进行的工作}

\subsection{系统调整}

内存大小:2048MB

显存大小:64MB

虚拟分配空间:128GB(动态分配差分存储)

\subsection{软件安装}

项目管理与编译:g++, gcc, make, cmake, automake;

编辑器:emacs, vim;

写文档和注释:texlive, texworks, doxygen;

计算:libboost, trilinos;

后处理:dx;

代码管理:git, ssh.

\subsection{配置}

软件源:http://mirrors.huaweicloud.com/repository/ubuntu

输入法:googlepinyin

共享文件夹:sf\_virtual

共享粘贴板:双向

\section{未来的工作}

\subsection{什么场合使用Linux}

对于我个人而言,使用Linux的场合包括但不限于学习。就近半年而言,我将会学一门名为“数据结构与算法的课程”,这门课程很可能需要Linux系统。对于其他的课程,我可能需要在Linux上进行文档编辑以完成作业任务。此外,我会在Linux上设计一些程序,以加深我对Linux的理解或完成学习任务\cite{https://www.jianshu.com/p/d3ac94fda9c2}。

\subsection{提升空间}

我目前的工作环境显然是不满足条件的。首先,我的笔记本电脑的电池不好,电池电量只能支持不到半小时的正常工作,一定程度上影响了我的学习;其次,我将虚拟机装在C盘,而现在C盘剩余容量不多。我计划暑假换一台笔记本电脑,并研究更好的Linux系统配置方式。特别地,我打算配置emacs以实现更加方便的文本编辑。

\section{稳定性与安全保障}

为了保证能够找回意外丢失的代码和资料,我打算借助github对我的代码进行托管,并将文件上传至网盘或其他能够云端保存文件的网站。我会时常对我的系统和一些重要文件进行备份,随时应对意外的发生。\cite{https://cloud.tencent.com/developer/article/1414120}\cite{https://www.zhihu.com/question/51309695}

\bibliographystyle{plain}
\bibliography{ref}

\end{document}
