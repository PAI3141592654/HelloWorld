\documentclass{ctexart}

\usepackage{graphicx}
\usepackage{amsmath}

\title{作业二}


\author{彭湃3210100061}

\begin{document}

\maketitle

\section{代码实践}
尝试对第三十页的try it out进行了一定的修改,加入了if和for语句,代码如下:

\begin{verbatim}
#!/bin/sh

salutation="Hello"
echo $salutation
echo "The program $0 is now running"
echo "The third parameter was $3"
echo "The second parameter was $2"
echo "The first parameter was $1"
echo "The parameter list was $*"
echo "The user's home directory is $HOME"

if [ $1 = $2 ]
then echo "The first two parameters are same"
else echo "The first two parameters are not same"
fi
    
echo "Please enter some new greetings"
read s1
echo $s1

for s in $s1
do
    echo "$s"
done


echo "The script is now complete"
exit 0
\end{verbatim}

输入案例1:
\begin{verbatim}
./test 你 你 皮 我瓜 > res.txt
撒日朗
cat res.txt

\end{verbatim}

输出如下:
\begin{verbatim}
Hello
The program ./test is now running
The third parameter was 皮
The second parameter was 你
The first parameter was 你
The parameter list was 你 你 皮 我瓜
The user's home directory is /home/pai314159
The first two parameters are same
Please enter some new greetings
撒日朗
撒日朗
The script is now complete
\end{verbatim}

输入案例2:
\begin{verbatim}
./test 你 ** 皮 我瓜 > res.txt
撒 日 朗
cat res.txt
\end{verbatim}

输出如下:
\begin{verbatim}
Hello
The program ./test is now running
The third parameter was test
The second parameter was res.txt
The first parameter was 你
The parameter list was 你 res.txt test test~ 皮 我瓜
The user's home directory is /home/pai314159
The first two parameters are not same
Please enter some new greetings
撒 日 朗
撒
日
朗
The script is now complete
\end{verbatim}

\section{分析}

\subsection{变量}

当我们定义了一个变量并对其赋值后,我们需要在变量名前加上\$以访问该变量,如代码中的\$salutation就是我们自己定义的变量.系统中有一些特殊的变量,例如\textbf{环境变量}、\textbf{参数变量}.代码中用到的环境变量有\$HOME(当前用户的家目录)、\$0(shell脚本的名字),用到的参数变量有\$1、\$2、\$3(脚本程序的第1、2、3个参数)、\$*(列出脚本程序的所有参数).

\textbf{rmk:}将变量赋值为*时,其将表示该目录下所有的文件名,每个文件名占据一个位置.案例2中的**与*等价.

\subsection{if\&for}

可用if语句进行判断,其中if后的中括号表示布尔判断命令,若为真则执行then后的内容,若不真则执行else后的内容,此外还可使用elif来添加新的条件.本程序使用if语句判断脚本程序的前两个参数是否相同并输出结果.

可用for语句来循环处理一组值.本程序使用for语句依次输出了用户输入的字符串.

\textbf{rmk:}若在read s1后加上s2,则案例2后面的输出将变成“撒(回车)撒”,因为此时“日”“朗”在s2中.


\end{document}
